\documentclass{beamer}
\usepackage[utf8]{inputenc}
\usepackage[brazil]{babel}
\usepackage{listings}

\graphicspath{{./imgs}}

\usetheme{Madrid}

\title{Workshop de Programação}
\author{Rafael \and Arthur \and Thomaz}
\institute{Prisma Game Lab}
\logo{\includegraphics[width=.05\textwidth]{logo}}
\date{\today}

\begin{document}
	\begin{frame}
		\titlepage
	\end{frame}

	\frame{\tableofcontents}

	\section{Programação Orientada a Objetos}
	\frame{\sectionpage}

	\subsection{Objeto}
	\begin{frame}
		\frametitle{O que é um objeto?}

		\begin{itemize}
			\item Um objeto é um elemento do programa que junta dados
				(atributos como variáveis) e ações sobre esses dados (métodos,
				funções do objeto).

			\item Objetos são muito utilizados para modelar elementos reais no
				código, de forma que a estrutura e funcionamento do programa se
				assemelham a situações do mundo real
		\end{itemize}

		\begin{block}{Exemplo}
			Em um jogo o jogador é normalmente representado por um objeto, que
			possui dados (atributos): vida e coordenadas. Assim como ações
			(métodos): levar dano e se mover.
		\end{block}
	\end{frame}

	\subsection{Classe} % mencionar construtores
	\begin{frame}
		\frametitle{Classe}

		\begin{itemize}
			\item Uma classe nada mais é do que a definição de um objeto, a
				planta que é usada para construir um objeto (uma instância da
				classe).

			\begin{definition}
				Uma "instância" de uma classe é um objeto criado a partir de
				tal classe. A classe por si só apenas define como o objeto é, e
				podemos então a usar para criar um objeto que podemos usar no
				programa.

				Por exemplo, a partir de uma planta (uma classe),
				podemos construir (instanciar) vários carros
				(instâncias/objetos) que por mais de terem a mesma estrutura,
				ainda são diferentes (ex. placas diferentes).
			\end{definition}

			\item Na classe, são definidos os atributos e métodos de um objeto.

			\item Para organização do código, é boa prática definir classes em
				arquivos separados no seu projeto, com o nome do arquivo sendo
				igual ao do objeto.
		\end{itemize}

	\end{frame}

	\begin{frame}[fragile]
		\begin{block}{Jogador.cs}
			\begin{lstlisting}[language=Java,basicstyle=\ttfamily,keywordstyle=\color{blue}]
public class Jogador
{
	private int vida;
	private float pos_x, pos_y;

	public void tomar_dano(int dano) {
		vida -= dano;
	}

	public void mover(float x, float y) {
		pos_x += x;
		pos_y += y;
	}
}
			\end{lstlisting}
		\end{block}
	\end{frame}

	\begin{frame}[fragile]
		\begin{block}{Instanciação}
			\begin{lstlisting}[language=Java,basicstyle=\ttfamily,keywordstyle=\color{blue}]
Jogador jogador1 = new Jogador();
Jogador jogador2 = new Jogador();
			\end{lstlisting}
		\end{block}
	\end{frame}

	\subsection{Atributos} % atributo estático
	\begin{frame}[fragile]
		\frametitle{Atributos}

		\begin{block}{Jogador.cs}
			\begin{lstlisting}[language=Java,basicstyle=\ttfamily,keywordstyle=\color{blue}]
private int vida;
private float pos_x, pos_y;
			\end{lstlisting}
		\end{block}

		\begin{itemize}
			\item Atributos são os dados definidos na classe do objeto.

			\item Atributos nada mais são do que variáveis definidas na classe.

			\item Valores de atributos de diferentes instâncias de uma classe,
				por padrão, são independentes uns dos outros.
		\end{itemize}
	\end{frame}

	\subsection{Métodos}
	\begin{frame}[fragile]
		\frametitle{Métodos}

		\begin{block}{Jogador.cs}
			\begin{lstlisting}[language=Java,basicstyle=\ttfamily,keywordstyle=\color{blue}]
public void tomar_dano(int dano) {
	vida -= dano;
}

public void mover(float x, float y) {
	pos_x += x;
	pos_y += y;
}
			\end{lstlisting}
		\end{block}

		\begin{itemize}
			\item Métodos representam ações de um objeto.

			\item São como funções mas que são definidas dentro de uma classe e
				são diretamente ligadas à uma instância.

		\end{itemize}
	\end{frame}

	\begin{frame}[fragile]
		\begin{block}{Chamada de método}
			\begin{lstlisting}[language=Java,basicstyle=\ttfamily,keywordstyle=\color{blue}]
jogador1.tomar_dano(5);
jogador2.mover(5.0f, 1.2f);
			\end{lstlisting}
		\end{block}
	\end{frame}

	\subsection{Encapsulamento} % boas práticas
	\begin{frame}[fragile]
		\frametitle{Encapsulamento}
		\begin{block}{Exemplo}
			\begin{lstlisting}[language=Java,basicstyle=\ttfamily,keywordstyle=\color{blue}]
public class Foo {}
private int bar;
public void baz();
			\end{lstlisting}
		\end{block}

		\begin{itemize}
			\item Podemos controlar quais partes do programa conseguem
				interagir com quais membros de um objeto.

			\item Membros públicos (public) podem ser acessados diretamente por
				código de fora da classe

			\item Membros privados (private) podem ser acessados apenas de
				dentro da própria classe.
		\end{itemize}

	\end{frame}

	\begin{frame}[fragile]
		\begin{alertblock}{Código inválido}
			\begin{lstlisting}[language=Java,basicstyle=\ttfamily,keywordstyle=\color{blue}]
jogador1.vida; // vida private
			\end{lstlisting}
		\end{alertblock}

		\begin{block}{Código válido}
			\begin{lstlisting}[language=Java,basicstyle=\ttfamily,keywordstyle=\color{blue}]
jogador1.mover(0.0f, 0.0f); // mover public
			\end{lstlisting}
		\end{block}
	\end{frame}

	\subsection{Herança} % Exemplo
	\begin{frame}
		\frametitle{Herança}

		\begin{itemize}
			\item Um dos grandes poderes da POO é construir novos objetos a
				partir de outros.

			\item Código que seria comum (atributos e/ou métodos comuns) entre
				classes pode ser posto em uma classe só, e outras classes podem
				"herdar" tais valores.
		\end{itemize}

		\begin{block}{Exemplo}
			Se no nosso jogo tivermos dois animais, por exemplo Gato e
			Cachorro, estes ainda tem muitas coisas em comum: Ambos de idade,
			nome, comem e dormem. Porém, o cachorro late e o gato mia.
		\end{block}
	\end{frame}

	\begin{frame}[fragile]
		\begin{block}{Exemplo}
			\begin{lstlisting}[language=Java,basicstyle=\ttfamily,keywordstyle=\color{blue}]
public class Animal {
	public int idade;
	public string nome;

	public void comer() {...}
	public void dormir() {...}
}

public class Gato : Animal {
	public void miar() {...}
}

public class Cachorro : Animal {
	public void latir() {...}
}
			\end{lstlisting}
		\end{block}
	\end{frame}

	\begin{frame}[fragile]
		\begin{block}{Exemplo}
			\begin{lstlisting}[language=Java,basicstyle=\ttfamily,keywordstyle=\color{blue}]
Gato gato = new Gato();
gato.miar();
gato.comer();
gato.dormir();
			\end{lstlisting}
		\end{block}
	\end{frame}

	\section{C\#}
	\frame{\sectionpage}

	\subsection{Tipagem e Variáveis} % conversão de tipos
	\begin{frame}[fragile]
		\frametitle{Tipagem e Variáveis}

		C\# é uma linguagem \textbf{tipada}, isso significa que precisamos
		dizer o tipo da informação que uma variável guarda. Isso é diferente de
		linguagens como Python, chamadas de \textbf{dinâmicas}.

		A principio tipagem pode parecer não ter vantagens sobre linguagens
		dinâmicas, porém tipos deixam mais claro o que uma variável guarda e
		impedem erros como passar o tipo errado para uma função.

		\begin{columns}
			\column{0.49\textwidth}
			\begin{block}{C\#}
				\begin{lstlisting}[language=Java,basicstyle=\ttfamily,keywordstyle=\color{blue}]
int numero = 1;
string nome = "Fulano";
float preco = 12.50f;
double altura = 1.8;
				\end{lstlisting}
			\end{block}
			\column{0.45\textwidth}
			\begin{block}{Python}
				\begin{lstlisting}[language=Python,basicstyle=\ttfamily,keywordstyle=\color{blue}]
numero = 1
nome = "Fulano"
preco = 12.50
altura = 1.8
				\end{lstlisting}
			\end{block}
		\end{columns}

	\end{frame}

	\begin{frame}[fragile]
		\frametitle{Conversão de tipos}

		Como C\# é uma linguagem tipada, se criarmos uma variável como int, esta
		apenas pode receber valores int. Se quisermos guardar um float em uma
		variável int, precisamos converter o valor para int.

		\begin{columns}
			\column{0.49\textwidth}
			\begin{block}{C\#}
				\begin{lstlisting}[language=Java,basicstyle=\ttfamily,keywordstyle=\color{blue}]
int numero = 1;
numero = (int)3.5f;
				\end{lstlisting}
			\end{block}
			\column{0.45\textwidth}
			\begin{block}{Python}
				\begin{lstlisting}[language=Python,basicstyle=\ttfamily,keywordstyle=\color{blue}]
numero = 1
numero = 3.5
				\end{lstlisting}
			\end{block}
		\end{columns}

		\begin{alertblock}{Conversão}
			Atenção, a conversão de float para int faz o número perder as suas
			casas decimais, então o valor em numero, no C\#, agora é 3.
		\end{alertblock}

	\end{frame}

	\begin{frame}
		\frametitle{Tipos comuns no C\#}

		\begin{description}
			\item[bool] Valor lógico booleano, true ou false.
			\item[int] Número inteiro de -2,147,483,648 à 2,147,483,647.
			\item[long] Número inteiro de -9,223,372,036,854,775,808 à
				9,223,372,036,854,775,807.
			\item[char] Letra única ('a', 'b', 'c', ...).
			\item[string] Cadeia de chars (letras), texto ("foo").
		\end{description}

		Objetos são tipos também! Portanto, se você criou um objeto "Jogador",
		Jogador é um tipo.
	\end{frame}

	\subsection{Aritmética}
	\begin{frame}
		\frametitle{Aritmética}

		Aritmética no C\# é bem simples e muito similar a Python.

		\begin{description}
			\item[$++a$] Incrementa valor de a.
			\item[$--a$] Decrementa valor de a.
			\item[a + b] Adição.
			\item[a - b] Subtração.
			\item[a * b] Multiplicação.
			\item[a / b] Divisão.
			\item[a \% b] Resto.
		\end{description}
	\end{frame}

	\subsection{Controle de Fluxo: Condicionais e Loops}
	\begin{frame}[fragile]
		\frametitle{if/else}

		No C\# podemos decidir caminhos para o código de acordo com condições.
		Assim como no Python, temos estruturas como o if/else.

		\begin{columns}
			\column{0.49\textwidth}
			\begin{block}{C\#}
				\begin{lstlisting}[language=Java,basicstyle=\ttfamily,keywordstyle=\color{blue}]
if (n > 0) {
	--n;
} else if (n == 0) {
	n = 1;
} else {
	++n;
}
				\end{lstlisting}
			\end{block}
			\column{0.45\textwidth}
			\begin{block}{Python}
				\begin{lstlisting}[language=Python,basicstyle=\ttfamily,keywordstyle=\color{blue}]
if n > 0:
	n -= 1
elif n == 0:
	n = 1
else:
	n += 1
				\end{lstlisting}
			\end{block}
		\end{columns}
	\end{frame}

	\begin{frame}[fragile]
		\frametitle{switch}

		No C\# também temos uma estrutura chamada "switch/case", que nos
		permite testar valores específicos (ou seja, igualdade).

		\begin{columns}
			\column{0.49\textwidth}
			\begin{block}{C\#}
				\begin{lstlisting}[language=Java,basicstyle=\ttfamily,keywordstyle=\color{blue}]
switch(n) {
	case 0:
		++m;
		break;
	case 1:
		--m;
		break;
	default:
		m = 0;
		break;
}
				\end{lstlisting}
			\end{block}
			\column{0.45\textwidth}
			\begin{block}{Python}
				\begin{lstlisting}[language=Python,basicstyle=\ttfamily,keywordstyle=\color{blue}]
if n == 0:
	m += 1
elif n == 1:
	m -= 1
else:
	m = 0
				\end{lstlisting}
			\end{block}
		\end{columns}
	\end{frame}

	\begin{frame}[fragile]
		\frametitle{while}

		Quando queremos repetir código até algo ser verdadeiro, podemos usar o
		while.

		Atenção que a condição é checada antes da execução do loop. Portanto,
		se a condição for falsa desde o começo, o loop não será executado nem
		uma vez.

		\begin{columns}
			\column{0.49\textwidth}
			\begin{block}{C\#}
				\begin{lstlisting}[language=Java,basicstyle=\ttfamily,keywordstyle=\color{blue}]
while (n > 2) {
	n /= 2;
}
				\end{lstlisting}
			\end{block}
			\column{0.45\textwidth}
			\begin{block}{Python}
				\begin{lstlisting}[language=Python,basicstyle=\ttfamily,keywordstyle=\color{blue}]
while n > 2:
	n /= 2
				\end{lstlisting}
			\end{block}
		\end{columns}
	\end{frame}

	\begin{frame}[fragile]
		\frametitle{do while}

		Se quisermos que a corpo do while seja executado pelo menos uma vez,
		podemos usar o do/while, que checa a condição apenas depois da execução
		de cada iteração.

		\begin{columns}
			\column{0.49\textwidth}
			\begin{block}{C\#}
				\begin{lstlisting}[language=Java,basicstyle=\ttfamily,keywordstyle=\color{blue}]
do {
	n /= 2;
} while (n > 2);
				\end{lstlisting}
			\end{block}
			\column{0.45\textwidth}
			\begin{block}{Python}
				\begin{lstlisting}[language=Python,basicstyle=\ttfamily,keywordstyle=\color{blue}]
n /= 2
while n > 2:
	n /= 2
				\end{lstlisting}
			\end{block}
		\end{columns}
	\end{frame}

	\begin{frame}[fragile]
		\frametitle{for loop}

		No C\# o for loop é bem diferente do Python. O for loop no C\# é mais versátil, e pode ser visto apenas como um "açúcar" sintático para o while.


			\begin{block}{for}
				\begin{lstlisting}[language=Java,basicstyle=\ttfamily,keywordstyle=\color{blue}]
for (int i = 0; i < 10; ++i) {
	// corpo
}
				\end{lstlisting}
			\end{block}
			\begin{block}{while}
				\begin{lstlisting}[language=Java,basicstyle=\ttfamily,keywordstyle=\color{blue}]
int i = 0;
while (i < 10) {
	// corpo
	++i;
}
				\end{lstlisting}
			\end{block}
	\end{frame}

	\subsection{Funções}
	\begin{frame}[fragile]
		\frametitle{Funções}
		Funções no C\#, assim como variáveis são tipadas. O tipo da função representa o tipo do seu retorno, sendo "void" o especial para funções que retornam nada.

			\begin{block}{Sintaxe da função}
				\begin{lstlisting}[language=Java,basicstyle=\ttfamily,keywordstyle=\color{blue}]
ecapsulamento tipo nome(parametros) {
	corpo
	return ...;
}
				\end{lstlisting}
			\end{block}
	\end{frame}

	\begin{frame}[fragile]
		\frametitle{Exemplo: Função que dobra int}
			\begin{block}{C\#}
				\begin{lstlisting}[language=Java,basicstyle=\ttfamily,keywordstyle=\color{blue}]
public int dobra(int n) {
	return n * 2;
}
				\end{lstlisting}
			\end{block}

			\begin{block}{Python}
				\begin{lstlisting}[language=Python,basicstyle=\ttfamily,keywordstyle=\color{blue}]
def dobra(n):
	return n * 2
				\end{lstlisting}
			\end{block}
	\end{frame}

	\subsection{Coleções: Arrays, Listas}
	\begin{frame}
		\frametitle{Coleções}

		No Python quando queremos guardar mais de um elemento em uma variável,
		usamos listas. Porém no C\# temos duas principais opções:

		\begin{description}
			\item[Array] Coleção de tamanho fixo, mas mais simples e mais eficiente
			\item[Lista] Coleção de tamanho dinâmico, mas um pouco mais complexa e menos eficiente
		\end{description}

	\end{frame}

	\begin{frame}[fragile]
		\frametitle{Array}

		Declaração de um array de 10 ints:
		\begin{lstlisting}[language=Java,basicstyle=\ttfamily,keywordstyle=\color{blue}]
int[] arr = new int[10];
		\end{lstlisting}

		Ler elemento:
		\begin{lstlisting}[language=Java,basicstyle=\ttfamily,keywordstyle=\color{blue}]
int n = arr[5];
		\end{lstlisting}

		Modificar elemento:
		\begin{lstlisting}[language=Java,basicstyle=\ttfamily,keywordstyle=\color{blue}]
arr[6] = n;
		\end{lstlisting}

	\end{frame}

	\begin{frame}[fragile]
		\frametitle{Lista}

		Criando uma lista de ints:
		\begin{lstlisting}[language=Java,basicstyle=\ttfamily,keywordstyle=\color{blue}]
List<int> lst = new List<int>();
		\end{lstlisting}

		Adicionar elemento:
		\begin{lstlisting}[language=Java,basicstyle=\ttfamily,keywordstyle=\color{blue}]
lst.Add(n);
		\end{lstlisting}

		Ler elemento:
		\begin{lstlisting}[language=Java,basicstyle=\ttfamily,keywordstyle=\color{blue}]
int n = lst[0];
		\end{lstlisting}

	\end{frame}

	\subsection{Eventos}
	\begin{frame}
		\frametitle{O que são Eventos}
		\begin{itemize}
			\item Forma de permitir que parte do programa avise outra parte do
				programa que algo aconteceu.
		\end{itemize}

		Exemplo de como acontece

		\begin{enumerate}
			\item Classe se inscreve em um evento, selecionando uma função que
				será chamada ao acontecer.

			\item Evento acontece, notificando todas as classes inscritas

			\item Classe recebe a notificação e a função inscrita é executada.

			\item Programa acaba ou Classe se desinscreve do evento.
		\end{enumerate}
	\end{frame}

	\begin{frame}[fragile]
		\frametitle{Delegate}

		\begin{itemize}
			\item No C\#, como tudo é tipado, precisamos dizer o "tipo" de
				funções que um evento aceita. Com isso, um "delegate" nada mais
				é do que um tipo que representa uma referência à um método.

			\item Podemos então tratar de métodos como se fossem "valores" e
				atribuí-los a variáveis e passar como argumentos para outras
				funções.
		\end{itemize}

		Delegates seguem o modelo:
		\begin{lstlisting}[language=Java,basicstyle=\ttfamily,keywordstyle=\color{blue}]
<encaps> delegate <retorno> nome(<parametros>)
		\end{lstlisting}

	\end{frame}

	\begin{frame}[fragile]
		\frametitle{Exemplo de Delegate}

		Digamos que queremos fazer uma função matemática que recebe uma
		operação (soma, subtração, divisão ou multiplicação), aplica sobre dois
		números e retorna o resultado. As operações definidas como:

		\begin{lstlisting}[language=Java,basicstyle=\ttfamily,keywordstyle=\color{blue}]
public int som(int a, int b) { return a + b; }
public int sub(int a, int b) { return a - b; }
public int mul(int a, int b) { return a * b; }
public int div(int a, int b) { return a / b; }
		\end{lstlisting}

		Podemos definir um "tipo" pra essas funções e então a função que recebe
		outra função e os números:

		\begin{lstlisting}[language=Java,basicstyle=\ttfamily,keywordstyle=\color{blue}]
public delegate int TipoOp(int x, int y);
public executarOp(TipoOp op, int a, int b) {
	return op(a, b);
}

executarOp(som, 5, 2) // -> 7
		\end{lstlisting}

	\end{frame}

	\begin{frame}
		\frametitle{Eventos}

		\begin{itemize}
			\item Digamos que sempre que nosso player leva dano, várias coisas
				precisam acontecer no jogo, por exemplo: Rodar uma animação e
				atualização a barra de vida.

			\item Poderíamos diretamente chamar tais ações, mas estariamos
				dando responsabilidades à classe player que não fazem muito
				sentido, e em sistemas mais complexos talvez deixe tudo mais
				dificil de atualizar

			\item Podemos então criar um evento que é disparado sempre que o
				player leva dano, e todas as classes interessadas podem se
				inscrever nesse evento. O Player então apenas precisará
				disparar o evento, sem saber quem está inscrito.
		\end{itemize}
	\end{frame}

	\begin{frame}
		\frametitle{Eventos}

		\begin{itemize}
			\item Cada Classe que se inscreverá terá um método que será chamado
				quando o evento acontecer. Esse método receberá por parâmetros
				informações relacionadas.

			\item Todos esses métodos precisam respeitar o delegate do evento,
				já que o evento chamará todas os métodos inscritos, e para isso
				precisa saber os tipos.

			\item Portanto precisamos definir um delegate(tipo) pro nosso
				evento, e criar o evento em si, para que outras classes possam
				se inscrever
		\end{itemize}

	\end{frame}

	\begin{frame}[fragile]
		\frametitle{Criação}

		Exemplo de uma função da classe da barra de vida:
		\begin{lstlisting}[language=Java,basicstyle=\ttfamily,keywordstyle=\color{blue}]
public void AtualizarBarra(int vida_nova) { ... }
		\end{lstlisting}

		Delegate do mesmo tipo:
		\begin{lstlisting}[language=Java,basicstyle=\ttfamily,keywordstyle=\color{blue}]
public delegate void PDamaged(int vida);
		\end{lstlisting}

		Definição de evento usando o delegate:
		\begin{lstlisting}[language=Java,basicstyle=\ttfamily,keywordstyle=\color{blue}]
public event PDamaged OnPlayerDamaged;
		\end{lstlisting}

		Inscrição em evento de uma Classe:
		\begin{lstlisting}[language=Java,basicstyle=\ttfamily,keywordstyle=\color{blue}]
jogador.PDamaged += AtualizarBarra;
		\end{lstlisting}

		Desinscrição em evento de uma Classe:
		\begin{lstlisting}[language=Java,basicstyle=\ttfamily,keywordstyle=\color{blue}]
jogador.PDamaged -= AtualizarBarra;
		\end{lstlisting}

	\end{frame}

	\begin{frame}[fragile]
		\begin{lstlisting}[language=Java,basicstyle=\ttfamily,keywordstyle=\color{blue}]
public class Jogador
{
    private int vida;
    private float pos_x, pos_y;

    public delegate void PDamaged(int vida);
    public event PDamaged OnPlayerDamaged;

    public void tomar_dano(int dano) {
        vida -= dano;
        OnPlayerDamaged?.Invoke(vida);
    }

    public void mover(float x, float y) {
        pos_x += x;
        pos_y += y;
    }
}
		\end{lstlisting}
	\end{frame}

	\section{Unity}
	\frame{\sectionpage}

	\subsection{Monobehaviour}
	\subsection{Input System}
	\subsection{Scriptable Objects (SOs)}
	\subsection{Corrotinas}
\end{document}

